\chapter{Primer} \label{chap:primer} % For referencing the chapter elsewhere, use \ref{Chapter1} 

Before particular strategies will be presented to combat the problem of capturing dispositions in social collectives, a primer on scale development is prefixed to prepare the upcoming chapters. It elaborates on general details, formalism, and some terminology (e.\,g., goals, scale development processes, and the latent variable logic).

An important goal in the social sciences is to assess social dispositions. Since almost all of them, like honor, are not directly observable, many fall into the category of latent dimensions. \textsc{Latent dimensions} are not directly observable variables. They lead to the \textsc{intricate problem} in the social sciences, namely that the OUI is hardly ever directly observable. Trying to cope with the intricate problem, dimensionality analysis takes the role of a means to infer latent dimensions. More precisely, it provides researchers with empirical arguments for their presence based on which inference is finally done. A crucial task in applied research is thus to capture the latent dimension, or, technically speaking, to quantify or measure it. This is where scales come in handy. \textsc{Scales} are measurement instruments trying to access latent dimensions with manifest (i.\,e., actually observable) variables. These are also known as \textsc{observable indicators}. The most common type of observable indicator in the social sciences is the questionnaire item (e.\,g., \enquote{You would praise a man who acts aggressively to an insult} -- strongly agree, somehow agree, $\dots$, somehow disagree, strongly disagree; \cite[see][]{Saucier2016}). Researchers often combine multiple ones on a scale to develop reliable instruments capable to assign each participant a numerical value which pins down his or her position on the underlying latent dimension. Overall, researchers receive a numerical outline of the social collective in form a distribution of participants' values, quantifying their object of interest.

Scales are the pivot point of dimensionality assessment. There are basically two entry points for researchers to develop them. In the first one, they start from concrete considerations of a latent dimension (e.\,g., honor) and invent questionnaire items that are reasonable to capture the latent dimension. For evaluation, data are collected, and the most appropriate items are selected to pin down participants' values on the underlying latent dimension. The second entry point starts with a data set and moves from exploration to measurement \parencite{DeLeeuw2005, Mair2015, Mair2018, McIver1981}. Thereby, item responses need to be explored for underlying patterns that structure the observable indicators before the participant's value on the underlying latent dimension is finally determined. In the following, the focus is only on the second scenario. So let's flesh it out hereafter.

Being confronted with a data set (${X}$)\endnote{The symbols are now introduced to smoothly prepare the formalism in the upcoming chapters. But any explanation can be understood without them. If they are inconvenient, just ignore them. The symbols will often allow concertizing the explanation, specifying the referenced element.}, which is actually a collection point for multiple content-related questions ($X_{i=1,\dots,m}$), the researcher has to explore the data set. Data exploration in dimensionality assessment means finding out about how many latent dimensions underlay the data. Often used synonymously is the expression \enquote{to analyze how the data are structured}. Either way, there are at least three plausible scenarios: First, there is no particular structure recognizable, which often means no latent dimension causes the observed item responses, or no latent dimension underlies the data. If they can be traced back to a single cause, one speaks of \textsc{unidimensionality}. multidimensionality, in turn, implies multiple latent dimensions which produce the observed item patterns. In later chapters, further distinction will be made between (ordinary) multidimensionality and \textsc{multiple unidimensionality}, but for now, it is more important to shed light on the assumed causal mechanism that interlinks indicator ($X_i$) and latent dimension ($\zeta$). In form of a question: Why do latent dimensions \textit{cause} the observed item response patterns? The answer can be given, following the \textsc{latent variable logic} \parencite{Wardrop1987}, which additionally clarifies the ultimate use of dimensionality analysis -- \textsc{latent variable inference}:

Let's assume there is a latent dimension ($\zeta$), which manifests in a series of observable indicators (${X}$). If so, each indicator ($X_i$) contains information on the latent dimension ($\zeta$) and thus is somehow related to it: $X_i \sim \zeta$. However, from the intricate problem one knows, as the latent dimension is unknown, so must be the relationship with it \parencite[ch. 8]{Bandalos2018}: $X_i \overset{?}{\sim} \zeta$. Nonetheless, all indicators arise from the same generative process -- they are produced by the same latent variable -- hence share common attributes. The common attributes, in turn, manifest in a particular structure ($\Sigma$) which includes the relationship $\rho$ among indicators ($X_i, X_j$):
\begin{equation}
\zeta \rightarrow \Sigma : \rho_{X_i, X_j}
\end{equation}
Correspondingly, the entire structure ($\Sigma \in {R}$) is like a measurable fingerprint to unlock the mystery of how many latent dimensions underlay the data. As a result, observed inter-item relations suffice to infer a latent dimension by assuming its presence \parencite{Thissen1989}. The handiest tool for dimensional assessment is thus an assembly point for inter-item relations: the correlation matrix (${R}$).

There is a huge number of possibilities to measure interrelations \parencite[ch. 5]{Adelson2019}, most common, however, is Pearson's correlation coefficient ($\rho$). Measuring the relationships between items via Person's coefficient, the correlation matrix (${R}$) is the place where all linear inter-item associations assemble. It is the foundation of dimensionality analysis because it is the spot where researchers get an inevitable (empirical) clue for latent dimensions,  inspecting relationships between observable indicators. It is the place where the latent dimensions (${\zeta}$) become visible, thus accessible, and, in the end, traceable. Therefore, the correlation matrix tags the starting point of most dimensional assessments \parencite[p. 159]{Gregory2014}. Unidimensional analysis (UDA), exploratory factor analysis (EFA) as well as exploratory Likert scaling (ELS) prove this to be true in the upcoming chapters.