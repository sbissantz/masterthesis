\chapter{Introduction} \label{chap:intro} % For referencing the chapter elsewhere, use \ref{Chapter1} 

One of the major bridges that connect various disciplines in the social sciences is widespread interest in multi-faceted constructs. They are part of research questions, utilized to develop theories, embedded in explanations, or used to describe even more complex social phenomena. To get clues of their existence and access them precisely is an essential part of common research practice in the social sciences. Take \enquote{honor} as an example. A whole research paradigm developed around the phenomenon, aiming to conceptualize and finally measure it \parencite{Beck1996, III1997, Nisbett2018, Saucier2016, Shackelford2005, VanOsch2013}. But there is nothing special to \enquote{honor}. The argument can be easily applied to other concepts. Take \enquote{pro-environmental behavior} for instance. Both examples poke right inside the intricate problems a social scientist has to confront: \textit{identify and measure complex social phenomena with respect to their own particularities}.

But what is actually the \textsc{object under investigation} (OUI) in social science, and what is peculiar to it? First, the OUI in the social sciences, things like \enquote{honor}, can be referred to as dispositions or latent dimensions. A \textsc{disposition} in social science is a (relatively) stable tendency to react. It does not primarily matter if the reaction manifests cognitively or leads to affective or behavioral consequences; but that the pattern is traceable and remains constant across a wide range of situations and circumstances. A broader range of typical examples includes areas such as abilities, skills, and expertise, as well as attitudes and personality traits. In other words, the main goal in applied social sciences research is to access and capture the (relatively) stable patterns that emerge from social interactions. Or, tying in with the above, the aim is to describe and explain dispositions in social collectives.

To answer the second, the particularity question, dispositions have special demands. They result from the OUI's inherent features -- its \textsc{characteristic dependencies}. Take \enquote{honor} as an example again. The extent to which honor is present in a society strongly hinges on the social collective; furthermore, it has turned out to be a time-dependent phenomenon \parencite{Nisbett2018, Shackelford2005, VanOsch2013}. This testifies to a special time and cultural dependency and induces a concept-specific continuity. Dispositions develop along a continuum, they can vary over time and manifest differently across cultures \parencite{Bornstein2018, Corballis1970, Nesselroade1972, Ram2015, Tisak1989, Tisak1990, Wickrama2020}. But they stand out through an additional feature. Dispositions have \textsc{blurry boundaries}, which means they share common attributes with other latent dimensions. They overlap with them, so to speak. Consequently, demarcating attributes is often a tough issue, because the boundaries between them are blurry. Think intuitively of \enquote{patriotism} and \enquote{nationalism}. With little elaboration, one will agree that they are not easily distinguishable \parencite[for an attempt, see][]{Costa2018}. Thus, both tend towards blurry boundaries.

A serious problem in applied research arises from neglecting the inherent properties of the OUI. Ignorance has shown to impose bias \parencite{Loo1979}. The so-called \textsc{ignorance bias} is often a result of bad (i.\,e., unreflected) default behavior. Methods are often applied without seeking for compatibility with the object under investigation. \textcite{Loo1979} conducted pioneer work, reviewing if the application of go-to methods in applied research coincides with the characteristic dependencies of the OUI. He found that often researcher's (default) choices smuggled in some set of assumptions that are  unreasonable under most circumstances (i.\,a., independence of latent dimensions). One could also say, researchers are blind to the characteristic features of the OUI from the background of their \textsc{bad defaults}. Think of the term \enquote{bad} here as a proxy for the incompatibility between the applied method and the characteristic features of the OUI. In this light, the shown \textsc{method-blindness fallacy} often results in \textsc{ignorance bias}.

Even though decent alternatives exist, less effort in the research community is devoted to reconciling statistical procedures with the characteristic features of the OUI. This thesis tries to fill in the gap. It is now time to eliminate bad defaults and replace them with recent and decent alternatives. The realignment will boost the quality of applied research and lead to a considerate upgrade of the modern researcher's toolbox. With respect to \textcite{Loo1979}'s findings, it is long overdue to scrutinize if the applied methods are appropriate for empirical research in the social sciences (i.\,e., in accordance with its OUI). Too long, it was ignored to seek compatibility between research methods and the object under investigation. However, this thesis sets out to address this issue. To attract and shift attention towards greater convergence of research methods and the OUI is the major concern of a more in-depth approach. More precisely, this requires (1) understanding to evaluate the most common statistical techniques, (2) reviewing their flaws and (3) assess their capability to identify latent dimensions, while keeping (4) sight of the OUI's characteristic features. Correspondingly, the features define the criterion for applicability of a method in this work. A subsequent critical reflection should recover sensitivity and contribute to developing new skills. Tools for exploratory purposes and theory development need to be tailored to the use in applied social science research. Ultimately, a series of bespoke tools will be proposed as decent alternatives. Replacing bad defaults will preempt method-blindness fallacy and additionally alleviate ignorance bias.

\subsection*{Outline}
Turning intentions into actions requires a more in-depth approach. Therefore, a robust methodological foundation building upon some intense thoughts on the object under investigation was already set up, \textit{before} now starting to work with it (ch. 1). Chapter 2 provides some requisite know-how for the chapters to come. The three upcoming chapters will be devoted to a particular type of dimensionality analysis, chapter 3 focuses on unidimensionality analysis and reliability analysis. Chapter 4, is about the common go-to method for multidimensionality analysis in the social sciences -- exploratory factor analysis. Solutions for the big three problems, namely the communality problem, the rotation problem, and the number of factor problem will be elaborated. A significant theoretical objection against multidimensional analysis finally opens up the opportunity to encounter a multiple unidimensional scaling approach (ch. 5). All chapters are accompanied by R code examples to facilitate practical application in self-instructions. The final chapter, chapter 6, summarizes and discusses previous key findings.