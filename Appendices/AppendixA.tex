% Appendix A

\chapter{Code to Reproduce the Graphics} % Main appendix title

\label{AppendixA} % For referencing this appendix elsewhere, use \ref{AppendixA}

\section*{Graphic 1}

\begin{verbnobox}[\tiny\arabic{VerbboxLineNo}\small\hspace{3ex}]
set.seed(359)
# Define factor loadings
loads <- rep(0.7, 12)
# Population correlation matrix (I)  
# tau <- psych::sim.congeneric(loads=loads)
# Sample correlation matrix 
R <- psych::sim.congeneric(loads=loads,N=50) 
# Visualize the correlation matrix
corrplot::corrplot.mixed(R, number.cex=.7)
\end{verbnobox}

\section*{Graphic 2}

\begin{verbnobox}[\tiny\arabic{VerbboxLineNo}\small\hspace{3ex}]
# Random number generator
set.seed(359)
# Define factor loadings
loads <- rep(0.6, 12)
# Population correlation matrix  
tau <- psych::sim.congeneric(loads=loads) 
# Sample correlation matrix 
tau_sample <- psych::sim.congeneric(loads=loads,N=50) 
# Rounded correlation matrix
round(tau_sample,2)
\end{verbnobox}

\section*{Graphic 3}

\begin{verbnobox}[\tiny\arabic{VerbboxLineNo}\small\hspace{3ex}]
# Random number generator
set.seed(359)
# Define factor loadings
loads <- rep(0.1, 12)
# Population correlation matrix (I)
# tau <- psych::sim.congeneric(loads=loads)
# Sample correlation matrix
R <- psych::sim.congeneric(loads=loads,N=50)
# Visualize the correlation matrix
corrplot::corrplot.mixed(R, number.cex=.7)
\end{verbnobox}

\section*{Graphic 4}

\begin{verbnobox}[\tiny\arabic{VerbboxLineNo}\small\hspace{3ex}]
# Random number generator
set.seed(0405)
# Determine 0-vector
zeros <- rep(0,12)
# Define a measurement model for x
# Note: values ~ loadings
fx <-matrix(c(.8, .8, .5, zeros, .7, .8, -.7, zeros,
              -.5, .7, -.8, zeros, .8, .8, .5), ncol = 4)
# Define the structure matrix
# On-diagonals: 1
phi <- diag(rep(1, 4))
# Between factor correlations
phi[1, 2] <- phi[2, 1] <- 0.2 ; phi[1, 3] <- phi[3, 1] <- 0.6
phi[1, 4] <- phi[4, 1] <- 0.7 ; phi[2, 3] <- phi[3, 2] <- 0.2
phi[2, 4] <- phi[4, 2] <- 0.6 ; phi[3, 4] <- phi[4, 3] <- 0.4
# Visualize the measurement model
# psych::structure.diagram(fx, phi, 
                  main = “Measurement model for X”)
# Produce the correlation matrix
R <- psych::sim.structure(fx, phi)$model
# Visualize the correlation matrix
corrplot::corrplot.mixed(R, number.cex=.7)
\end{verbnobox}

\section*{Graphic 5}

\begin{verbnobox}[\tiny\arabic{VerbboxLineNo}\small\hspace{3ex}]
# Random number generator
set.seed(213)
# Determine 0-vector
N <- 20
loadings <- sample(seq(-.8,.8,0.01), N, replace=TRUE)
# Define a measurement model for x
# Note: values ~ loadings
fx <-matrix(loadings, ncol = 2)
# Define the structure matrix
# On-diagonals: 1
phi <- diag(rep(1, 2))
# Between factor correlations
phi[1, 2] <- phi[2, 1] <- -0.2
# Visualize the measurement model
# psych::structure.diagram(fx, phi,
                  main = “Measurement model for X”)
# Produce the correlation matrix
R <- psych::sim.structure(fx, phi)$model
# Visualize the correlation matrix
#corrplot::corrplot.mixed(R, number.cex=.7)
fa <- psych::fa(R, nfactors = 2)
plot(NULL, xlab = “honor”,  ylab = “pride”,
     ylim = c(-1,1), xlim = c(-1,1))
points(fa$loadings[,1], fa$loadings[,2], pch=20, cex=.7)
text(fa$loadings[,1], fa$loadings[,2]+.1,
     paste0(“I”, 2:(N/2)+1), cex = .7)
l1 <- .5 ; l2 <- .6
points(l1, l2, pch=20, cex=.7)
text(l1, l2+.1, “I:(0.5,0.6)”, cex = .6, col="red")
lines(c(l1,l1), c(0,l2) ,lty=2, lwd=.5)
lines(c(0,l1), c(l2,l2) ,lty=2, lwd=.5)
abline(h = 0, v = 0)
\end{verbnobox}

\section*{Graphic 6}

\begin{verbnobox}[\tiny\arabic{VerbboxLineNo}\small\hspace{3ex}]
# psych::structure.diagram(fx, phi, 
                  main = "Measurement model for X")
# Produce the correlation matrix
R <- psych::sim.structure(fx, phi)$model
# Visualize the correlation matrix
#corrplot::corrplot.mixed(R, number.cex=.7)
fa <- psych::fa(R, nfactors = 2)
plot(NULL, xlab = \u201chonor\u201d,  ylab = "pride",
     ylim = c(-1,1), xlim = c(-1,1))
points(fa$loadings[,1], fa$loadings[,2], pch=20, cex=.7)
text(fa$loadings[,1], fa$loadings[,2]+.1,
     paste0(\u201cI\u201d, 2:(N/2)+1), cex = .7)
l1 <- .5 ; l2 <- .6
points(l1, l2, pch=20, cex=.7)
text(l1, l2+.1, \u201cI:(0.5,0.6)\u201d, cex = .6, col="red")
lines(c(l1,l1), c(0,l2) ,lty=2, lwd=.5)
lines(c(0,l1), c(l2,l2) ,lty=2, lwd=.5)
abline(h = 0, v = 0)
\end{verbnobox}

\section*{Graphic 7}

\begin{verbnobox}[\tiny\arabic{VerbboxLineNo}\small\hspace{3ex}]
# Random number generator
set.seed(213)
# Determine number of items
N <- 20
loadings <- sample(seq(-.8,.8,0.01), N, replace=TRUE)
# Define a measurement model for x
# Note: values ~ loadings
fx <-matrix(loadings, ncol = 2)
# Define the structure matrix
# On-diagonals: 1
phi <- diag(rep(1, 2))
# Between factor correlations
phi[1, 2] <- phi[2, 1] <- 0.6
# Visualize the measurement model
# psych::structure.diagram(fx, phi, 
                  main = “Measurement model for X”)
# Produce the correlation matrix
R <- psych::sim.structure(fx, phi)$model
# Scree plot
psych::scree(R, factors = FALSE)
\end{verbnobox}

\section*{Grafik 8}

\begin{verbnobox}[\tiny\arabic{VerbboxLineNo}\small\hspace{3ex}]
# Random number generator
set.seed(213)
# Determine the number of items
N <- 20
# Specify the loading pattern
loadings <- sample(seq(-.8,.8,0.01), N, replace=TRUE)
# Define a measurement model for x
# Note: values ~ loadings
fx <-matrix(loadings, ncol = 2)
# Define the structure matrix
# On-diagonals: 1
phi <- diag(rep(1, 2))
# Between factor correlations
phi[1, 2] <- phi[2, 1] <- 0.2
# Visualize the measurement model
# psych::structure.diagram(fx, phi, 
                  main = “Measurement model for X”)
# Produce the correlation matrix
X <- psych::sim.structure(fx, phi, n=50)$observed
# Parallel analysis
paran::paran(X, quietly = TRUE, graph = TRUE, cfa = FALSE)
\end{verbnobox}

\section*{Grafik 9}

\begin{verbnobox}[\tiny\arabic{VerbboxLineNo}\small\hspace{3ex}]
# Random number generator
set.seed(213)
# Determine the number of items
N <- 20
# Specify a loading pattern
loadings <- sample(seq(-.8,.8,0.01), N, replace=TRUE)
# Define a measurement model for x
# Note: values ~ loadings
fx <-matrix(loadings, ncol = 2)
# Define the structure matrix
# On-diagonals: 1
phi <- diag(rep(1, 2))
# Between factor correlations
phi[1, 2] <- phi[2, 1] <- 0.6
# Visualize the measurement model
# psych::structure.diagram(fx, phi, 
                  main = “Measurement model for X”)
# Produce the correlation matrix
X <- psych::sim.structure(fx, phi, n=100)$observed
# Scree plot
EFA.MRFA::hullEFA(X, extr=”ML",display = FALSE, graph = TRUE)
\end{verbnobox}
